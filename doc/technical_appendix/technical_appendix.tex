\documentclass[a4paper,UKenglish,cleveref, autoref, thm-restate]{lipics-v2021}
%This is a template for producing LIPIcs articles. 
%See lipics-v2021-authors-guidelines.pdf for further information.
%for A4 paper format use option "a4paper", for US-letter use option "letterpaper"
%for british hyphenation rules use option "UKenglish", for american hyphenation rules use option "USenglish"
%for section-numbered lemmas etc., use "numberwithinsect"
%for enabling cleveref support, use "cleveref"
%for enabling autoref support, use "autoref"
%for anonymousing the authors (e.g. for double-blind review), add "anonymous"
%for enabling thm-restate support, use "thm-restate"
%for enabling a two-column layout for the author/affilation part (only applicable for > 6 authors), use "authorcolumns"
%for producing a PDF according the PDF/A standard, add "pdfa"

\pdfoutput=1 %uncomment to ensure pdflatex processing (mandatatory e.g. to submit to arXiv)
\hideLIPIcs  %uncomment to remove references to LIPIcs series (logo, DOI, ...), e.g. when preparing a pre-final version to be uploaded to arXiv or another public repository

%\graphicspath{{./graphics/}}%helpful if your graphic files are in another directory

\bibliographystyle{plainurl}% the mandatory bibstyle

\title{Towards Practical First-Order Model Counting: Technical Appendix}

\author{Ananth K. Kidambi}{Indian Institute of Technology Bombay, Mumbai, India}{210051002@iitb.ac.in}{}{}
\author{Guramrit Singh}{Indian Institute of Technology Bombay, Mumbai, India}{guramrit@iitb.ac.in}{}{}
\author{Paulius Dilkas}{University of Toronto, Toronto, Canada \and Vector Institute, Toronto, Canada \and \url{https://dilkas.github.io/}}{paulius.dilkas@utoronto.ca}{https://orcid.org/0000-0001-9185-7840}{}
\author{Kuldeep S. Meel}{University of Toronto, Toronto, Canada \and \url{https://www.cs.toronto.edu/~meel/}}{meel@cs.toronto.edu}{https://orcid.org/0000-0001-9423-5270}{}

\authorrunning{A.\,K. Kidambi, G. Singh, P. Dilkas, and K.\,S. Meel}
\Copyright{Ananth K. Kidambi, Guramrit Singh, Paulius Dilkas, and Kuldeep S. Meel}

\relatedversion{} %optional, e.g. full version hosted on arXiv, HAL, or other respository/website

%\supplement{}%optional, e.g. related research data, source code, ... hosted on a repository like zenodo, figshare, GitHub, ...
%\supplementdetails[linktext={opt. text shown instead of the URL}, cite=DBLP:books/mk/GrayR93, subcategory={Description, Subcategory}, swhid={Software Heritage Identifier}]{General Classification (e.g. Software, Dataset, Model, ...)}{URL to related version} %linktext, cite, and subcategory are optional

%\funding{(Optional) general funding statement \dots}%optional, to capture a funding statement, which applies to all authors. Please enter author specific funding statements as fifth argument of the \author macro.

%\acknowledgements{I want to thank \dots}%optional

%\nolinenumbers %uncomment to disable line numbering



%Editor-only macros:: begin (do not touch as author)%%%%%%%%%%%%%%%%%%%%%%%%%%%%%%%%%%
\EventEditors{John Q. Open and Joan R. Access}
\EventNoEds{2}
\EventLongTitle{42nd Conference on Very Important Topics (CVIT 2016)}
\EventShortTitle{CVIT 2016}
\EventAcronym{CVIT}
\EventYear{2016}
\EventDate{December 24--27, 2016}
\EventLocation{Little Whinging, United Kingdom}
\EventLogo{}
\SeriesVolume{42}
\ArticleNo{23}
%%%%%%%%%%%%%%%%%%%%%%%%%%%%%%%%%%%%%%%%%%%%%%%%%%%%%%

\usepackage{booktabs}
\usepackage{pifont}
\usepackage[backgroundcolor=lightgray]{todonotes} % TODO: temporary

\newcommand{\cmark}{\ding{51}}
\newcommand{\xmark}{\ding{55}}
\newcommand{\Ctwo}{$\mathsf{C}^{2}$}
\newcommand{\FO}{$\mathsf{FO}$}
\newcommand{\UFO}{$\mathsf{UFO}^{2} + \mathsf{CC}$}
\newcommand{\Cranetwo}{\textsc{Gantry}}
\newcommand{\friends}{\emph{Friends \& Smokers}}
\newcommand{\functions}{\emph{Functions}}
\newcommand{\bijections}{\emph{Bijections}}

\crefalias{formula}{equation}
\crefname{formula}{Sentence}{Sentences}
\creflabelformat{formula}{#2\textup{(#1)}#3}

\begin{document}

\maketitle

\section{The Three Logics of FOMC}\label{sec:threelogics}

\begin{table}[t]
  \centering
  \begin{tabular}{llclll}
    \toprule
    Logic & Sorts & Constants & Variables & Quantifiers & Additional atoms\\
    \midrule
    \FO & one or more & \cmark & unlimited & $\forall$, $\exists$ & $x = y$\\
    \Ctwo & one & \xmark & two & $\forall$, $\exists$, $\exists^{= k}$, $\exists^{\le k}$, $\exists^{\ge k}$ & ---\\
    \UFO & one & \xmark & two & $\forall$ & $|P| = m$\\
    \bottomrule
  \end{tabular}
  \caption{A comparison of the three logics used in FOMC\@. The
    2\textsuperscript{nd}--5\textsuperscript{th} columns refer to: the number of
    sorts, support for constants, the maximum number of variables, and supported
    quantifiers, respectively. The last column lists supported atoms in addition
    to those of the form $P(\mathbf{t})$ for a predicate $P/n$ and an $n$-tuple
    of terms $\mathbf{t}$. Here: $k$ and $m$ are non-negative integers, with the
    latter depending on the domain size, $P$ represents a predicate, and $x$ and
    $y$ are terms. }\label{tbl:logics}
\end{table}

\todo[inline,caption={}]{
  \begin{itemize}
    \item Reference the table
    \item Make sure the rest of the paper doesn't refer to these logics (or
          describe examples in them)
    \item Move the URLs to the main text (and make sure that each algorithm name
          is combined with a citation at least once)
    \item Later (probably) move this to a different file
  \end{itemize}
}

\renewcommand*{\thefootnote}{\fnsymbol{footnote}}

There are three first-order logics commonly used in FOMC: \FO{}, \Ctwo{}, and
\UFO{}. First, \FO{} is the input format for
\textsc{ForcLift}\footnote{\url{https://github.com/UCLA-StarAI/Forclift}} and
its extensions
\textsc{Crane}\footnote{\url{https://doi.org/10.5281/zenodo.8004077}} and
\Cranetwo{}. Second, \Ctwo{} is often used in the literature on
\textsc{FastWFOMC}\footnote{\url{https://github.com/jan-toth/FastWFOMC.jl}} and
related methods~\cite{DBLP:journals/jair/Kuzelka21,DBLP:conf/aaai/MalhotraS22}.
Finally, \UFO{} is the input format supported by the most recent implementation
of \textsc{FastWFOMC}~\cite{DBLP:conf/kr/TothK24}. All three logics are
function-free, and domains are always assumed to be finite. As usual, we
presuppose the \emph{unique name assumption}, which states that two constants
are equal if and only if they are the same constant~\cite{DBLP:books/aw/RN2020}.

\renewcommand*{\thefootnote}{\arabic{footnote}}

In \FO{}, each term is assigned to a \emph{sort}, and each predicate $P/n$ is
assigned to a sequence of $n$ sorts. Each sort has its corresponding domain.
These assignments to sorts are typically left implicit and can be reconstructed
from the quantifiers. For example, $\forall x,y \in \Delta\text{. }P(x, y)$
implies that variables $x$ and $y$ have the same sort. On the other hand,
$\forall x \in \Delta\text{. }\forall y \in \Gamma\text{. } P(x, y)$ implies
that $x$ and $y$ have different sorts, and it would be improper to write, for
example, $\forall x \in \Delta\text{. }\forall y \in \Gamma\text{.
} P(x, y) \lor x = y$. \FO{} is also the only logic to support constants,
sentences with more than two variables, and the equality predicate. While we do
not explicitly refer to sorts in subsequent sections of this paper, the
many-sorted nature of \FO{} is paramount to the algorithms presented therein.

\begin{remark*}
  In the case of \textsc{ForcLift} and its extensions, support for a sentence as
  valid input does not imply that the algorithm can compile the sentence into a
  circuit or graph suitable for lifted model counting. However, it is known that
  \textsc{ForcLift} compilation is guaranteed to succeed on any \FO{} sentence
  without constants and with at most two
  variables~\cite{DBLP:conf/nips/Broeck11,DBLP:conf/kr/BroeckMD14}.
\end{remark*}

Compared to \FO{}, \Ctwo{} and \UFO{} lack support for constants, the equality
predicate, multiple domains, and sentences with more than two variables. The
advantage that \Ctwo{} brings over \FO{} is the inclusion of \emph{counting
  quantifiers}. That is, alongside $\forall$ and $\exists$, \Ctwo{} supports
$\exists^{=k}$, $\exists^{\le k}$, and $\exists^{\ge k}$ for any positive
integer $k$. For example, $\exists^{=1} x\text{. }\phi(x)$ means that there
exists \emph{exactly one} $x$ such that $\phi(x)$, and $\exists^{\le 2} x\text{.
}\phi(x)$ means that there exist \emph{at most two} such $x$. \UFO{}, on the
other hand, does not support any existential quantifiers but instead
incorporates \emph{(equality) cardinality constraints}. For example, $|P| = 3$
constrains all models to have \emph{precisely three positive literals with the
  predicate $P$}.

\subsection{Our Benchmarks in \Ctwo{} and \UFO{}}

\todo[inline,caption={}]{
  \begin{itemize}
    \item An introductory paragraph
    \item Figure out the right order in which the information below should be
          presented
    \item Make the text below coherent
    \item Refer to it in the main text (and make sure the reference sticks when
          I transfer this to a separate file)
  \end{itemize}
}

\friends{} in \Ctwo{} and \UFO{} is the same as in \FO{}.

For \bijections{}, the equivalent sentence in \Ctwo{} is
\[
  (\forall x \in \Delta\text{. }\exists^{=1} y \in \Delta\text{.
  }P(x, y)) \land (\forall y \in \Delta\text{. }\exists^{=1} x \in \Delta\text{.
  }P(x, y)).
\]
Similarly, in \UFO{} the same sentence can be written as
\[
  (\forall x, y \in \Delta\text{.
  }R(x) \lor \neg P(x, y)) \land (\forall x, y \in \Delta\text{.
  }S(x) \lor \neg P(y, x)) \land (|P| = |\Delta|),
\]
where $w^{-}(R) = w^{-}(S) = -1$.

For \functions{}, in \Ctwo{} one would write $\forall x \in \Delta\text{.
}\exists^{=1} y \in \Delta\text{. }P(x, y)$. In \UFO{}, the same could be
written as
\begin{equation}\label[formula]{eq:functions1}
  (\forall x, y \in \Delta\text{. }S(x) \lor \neg P(x, y)) \land (|P| = |\Delta|),
\end{equation}
where $w^{-}(S) = -1$. Although \cref{eq:functions1} has more models compared to
its counterpart in \Ctwo{}, the negative weight $w^{-}(S) = -1$ makes some of
the terms in the definition of WFOMC cancel out.

\bibliography{../paper/paper.bib}

\end{document}
