\documentclass[a4paper,UKenglish,cleveref, autoref, thm-restate]{lipics-v2021}
\pdfoutput=1
\hideLIPIcs
\bibliographystyle{plainurl}

\title{Towards Practical First-Order Model Counting: Technical Appendix}

\author{Ananth K. Kidambi}{Indian Institute of Technology Bombay, Mumbai, India}{210051002@iitb.ac.in}{}{}
\author{Guramrit Singh}{Indian Institute of Technology Bombay, Mumbai, India}{guramrit@iitb.ac.in}{}{}
\author{Paulius Dilkas}{University of Toronto, Toronto, Canada \and Vector Institute, Toronto, Canada \and \url{https://dilkas.github.io/}}{paulius.dilkas@utoronto.ca}{https://orcid.org/0000-0001-9185-7840}{}
\author{Kuldeep S. Meel}{University of Toronto, Toronto, Canada \and \url{https://www.cs.toronto.edu/~meel/}}{meel@cs.toronto.edu}{https://orcid.org/0000-0001-9423-5270}{}

\authorrunning{A.\,K. Kidambi, G. Singh, P. Dilkas, and K.\,S. Meel}
\Copyright{Ananth K. Kidambi, Guramrit Singh, Paulius Dilkas, and Kuldeep S. Meel}

\relatedversion{}

\EventEditors{John Q. Open and Joan R. Access}
\EventNoEds{2}
\EventLongTitle{42nd Conference on Very Important Topics (CVIT 2016)}
\EventShortTitle{CVIT 2016}
\EventAcronym{CVIT}
\EventYear{2016}
\EventDate{December 24--27, 2016}
\EventLocation{Little Whinging, United Kingdom}
\EventLogo{}
\SeriesVolume{42}
\ArticleNo{23}

\usepackage{booktabs}
\usepackage{pifont}

\newcommand{\cmark}{\ding{51}}
\newcommand{\xmark}{\ding{55}}
\newcommand{\Ctwo}{$\mathsf{C}^{2}$}
\newcommand{\FO}{$\mathsf{FO}$}
\newcommand{\UFO}{$\mathsf{UFO}^{2} + \mathsf{CC}$}
\newcommand{\Cranetwo}{\textsc{Gantry}}
\newcommand{\friends}{\emph{Friends \& Smokers}}
\newcommand{\functions}{\emph{Functions}}
\newcommand{\bijections}{\emph{Bijections}}

\crefalias{formula}{equation}
\crefname{formula}{sentence}{sentences}
\creflabelformat{formula}{#2\textup{(#1)}#3}

\begin{document}

\maketitle

\section{The Three Logics of FOMC}\label{sec:threelogics}

\begin{table}[t]
  \centering
  \begin{tabular}{llclll}
    \toprule
    Logic & Sorts & Constants & Variables & Quantifiers & Additional atoms\\
    \midrule
    \FO & one or more & \cmark & unlimited & $\forall$, $\exists$ & $x = y$\\
    \Ctwo & one & \xmark & two & $\forall$, $\exists$, $\exists^{= k}$, $\exists^{\le k}$, $\exists^{\ge k}$ & ---\\
    \UFO & one & \xmark & two & $\forall$ & $|P| = m$\\
    \bottomrule
  \end{tabular}
  \caption{A comparison of the three logics used in FOMC\@. The
    2\textsuperscript{nd}--5\textsuperscript{th} columns refer to the number of
    sorts, support for constants, the maximum number of variables, and supported
    quantifiers, respectively. The last column lists supported atoms in addition
    to those of the form $P(\mathbf{t})$ for a predicate $P/n$ and an $n$-tuple
    of terms $\mathbf{t}$. Here, $k$ and $m$ are non-negative integers, where
    $m$ depends on the domain size, $P$ is a predicate, and $x$ and $y$ are
    terms.}\label{tbl:logics}
\end{table}

FOMC commonly utilises three types of first-order logic: \FO{}, \Ctwo{}, and
\UFO{}. \Cref{tbl:logics} summarises the key differences among them. \FO{} is
the input format for \textsc{ForcLift} and its extensions \textsc{Crane} and
\Cranetwo{}. \Ctwo{} is often used in the literature on \textsc{FastWFOMC} and
related methods~\cite{DBLP:journals/jair/Kuzelka21,DBLP:conf/aaai/MalhotraS22}.
(Note that no algorithm accepts \Ctwo{} as input.) Finally, \UFO{} is the input
format supported by the most recent implementation of
\textsc{FastWFOMC}~\cite{DBLP:conf/kr/TothK24}. All three logics are
function-free, and domains are always assumed to be finite. As usual, we
presuppose the \emph{unique name assumption}, which states that two constants
are equal if and only if they are the same constant~\cite{DBLP:books/aw/RN2020}.

In \FO{}, each term has a designated \emph{sort}, and each predicate $P/n$
corresponds to a sequence of $n$ sorts. Each sort has its corresponding domain.
These assignments to sorts are typically left implicit and follow from the
quantifiers, e.g., $\forall x,y \in \Delta\text{. }P(x, y)$ implies that the
variables $x$ and $y$ have the same sort. On the other hand,
$\forall x \in \Delta\text{. }\forall y \in \Gamma\text{. } P(x, y)$ implies
that $x$ and $y$ have different sorts, and it would be improper to write, for
example, $\forall x \in \Delta\text{. }\forall y \in \Gamma\text{.
} P(x, y) \lor x = y$. \FO{} is also the only logic to support constants,
sentences with more than two variables, and the equality predicate. While we do
not explicitly refer to sorts in the paper, the many-sorted nature of \FO{} is
paramount to the algorithms presented therein.

\begin{remark*}
  In the case of \textsc{ForcLift} and its extensions, support for a sentence as
  valid input does not imply that the algorithm can compile the sentence into a
  circuit or graph suitable for lifted model counting. However,
  \textsc{ForcLift} compilation always succeeds on any \FO{} sentence without
  constants and with at most two
  variables~\cite{DBLP:conf/nips/Broeck11,DBLP:conf/kr/BroeckMD14}.
\end{remark*}

Compared to \FO{}, \Ctwo{} and \UFO{} lack support for constants, the equality
predicate, multiple domains, and sentences with more than two variables. The
advantage that \Ctwo{} brings over \FO{} is the inclusion of \emph{counting
  quantifiers}. That is, alongside $\forall$ and $\exists$, \Ctwo{} supports
$\exists^{=k}$, $\exists^{\le k}$, and $\exists^{\ge k}$ for any positive
integer $k$. For example, $\exists^{=1} x\text{. }\phi(x)$ means that there
exists \emph{exactly one} $x$ such that $\phi(x)$, and $\exists^{\le 2} x\text{.
}\phi(x)$ means that there exist \emph{at most two} such $x$. \UFO{}, on the
other hand, does not support any existential quantifiers but instead
incorporates \emph{(equality) cardinality constraints}. For example, $|P| = 3$
constrains all models to have \emph{precisely three positive literals with the
  predicate $P$}.

\paragraph*{Our Benchmarks in \Ctwo{} and \UFO{}}
For completeness and reproducibility, let us translate the benchmark sentences
from \FO{} to \Ctwo{} and \UFO{}. Since \friends{} is a relatively simple
sentence, it remains the same in \Ctwo{} and \UFO{}. For \functions{}, in
\Ctwo{}, one would write
\[
  \forall x \in \Delta\text{. }\exists^{=1} y \in \Delta\text{. }P(x, y).
\]
In \UFO{}, the equivalent formulation is
\begin{equation}\label[formula]{eq:functions1}
  (\forall x, y \in \Delta\text{. }S(x) \lor \neg P(x, y)) \land (|P| = |\Delta|),
\end{equation}
where $w^{-}(S) = -1$. Although \cref{eq:functions1} has more models than its
counterpart in \Ctwo{}, the negative weight $w^{-}(S) = -1$ causes some of the
terms in the definition of WFOMC to cancel out. The translation of \bijections{}
is similar to that of \functions{}. In \Ctwo{}, one could write
\[
  (\forall x \in \Delta\text{. }\exists^{=1} y \in \Delta\text{.
  }P(x, y)) \land (\forall y \in \Delta\text{. }\exists^{=1} x \in \Delta\text{.
  }P(x, y)).
\]
Similarly, in \UFO{}, the equivalent formulation is
\[
  (\forall x, y \in \Delta\text{.
  }R(x) \lor \neg P(x, y)) \land (\forall x, y \in \Delta\text{.
  }S(x) \lor \neg P(y, x)) \land (|P| = |\Delta|),
\]
where $w^{-}(R) = w^{-}(S) = -1$.

\bibliography{../paper/paper.bib}

\end{document}
