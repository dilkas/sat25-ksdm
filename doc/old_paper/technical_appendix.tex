%File: anonymous-submission-latex-2025.tex
\documentclass[letterpaper]{article} % DO NOT CHANGE THIS
\usepackage[submission]{aaai25}  % DO NOT CHANGE THIS
\usepackage{times}  % DO NOT CHANGE THIS
\usepackage{helvet}  % DO NOT CHANGE THIS
\usepackage{courier}  % DO NOT CHANGE THIS
\usepackage[hyphens]{url}  % DO NOT CHANGE THIS
\usepackage{graphicx} % DO NOT CHANGE THIS
\urlstyle{rm} % DO NOT CHANGE THIS
\def\UrlFont{\rm}  % DO NOT CHANGE THIS
\usepackage{natbib}  % DO NOT CHANGE THIS AND DO NOT ADD ANY OPTIONS TO IT
\usepackage{caption} % DO NOT CHANGE THIS AND DO NOT ADD ANY OPTIONS TO IT
\frenchspacing  % DO NOT CHANGE THIS
\setlength{\pdfpagewidth}{8.5in} % DO NOT CHANGE THIS
\setlength{\pdfpageheight}{11in} % DO NOT CHANGE THIS

\usepackage[linesnumbered,ruled,vlined]{algorithm2e}
\usepackage{amsfonts}
\usepackage{amsmath}
\usepackage{amsthm}
\usepackage[capitalize,noabbrev]{cleveref}

% Keep the \pdfinfo as shown here. There's no need
% for you to add the /Title and /Author tags.
\pdfinfo{
/TemplateVersion (2025.1)
}

\setcounter{secnumdepth}{2} %May be changed to 1 or 2 if section numbers are desired.

\SetKwFunction{CompileWithBaseCases}{CompileWithBaseCases}
\SetKwFunction{FindBaseCases}{FindBaseCases}

\newcommand{\Cranetwo}{\textsc{Crane2}}
\newcommand{\expr}{\mathtt{expr}}

\newtheorem{theorem}{Theorem}
\newtheorem{assumption}{Assumption}
\newtheorem{corollary}{Corollary}
\newtheorem{fact}{Observation}
\newtheorem{lemma}{Lemma}

\title{Towards Practical First-Order Model Counting: Technical Appendix}
\author{Anonymous Submission}
\affiliations{}

\begin{document}

\maketitle

\section{The Proof of Theorem 1}

To demonstrate that the base cases identified by \FindBaseCases are sufficient,
we begin with a few observations that stem from
previous~\cite{DBLP:conf/ijcai/BroeckTMDR11,DBLP:conf/kr/DilkasB23} and this
work. Let $\mathcal{E}$ represent the equations returned by
\CompileWithBaseCases.

\begin{fact}\label{assumption1}
  For each function $f$, there is precisely one equation $e \in \mathcal{E}$
  with $f(\mathbf{x})$ on the left-hand side where all $x_{i}$'s are variables
  (i.e., $e$ is not a base case). We refer to $e$ as the \emph{definition} of
  $f$.
\end{fact}

\begin{fact}\label{assumption2}
  There is a \emph{topological ordering} of all functions ${(f_{i})}_{i}$ in
  $\mathcal{E}$ such that equations in $\mathcal{E}$ with $f_{i}$ on the
  left-hand side do not contain function calls to $f_{j}$ with $j > i$. This
  condition prevents mutual recursion and other cyclic scenarios.
\end{fact}

\begin{fact}\label{assumption3}
  For every equation $(f(\mathbf{x}) = \expr) \in \mathcal{E}$, the evaluation
  of $\expr$ terminates when provided with the values of all relevant function
  calls.
\end{fact}

\begin{corollary}\label{fact}
  If $f$ is a non-recursive function with no function calls on the right-hand
  side of its definition, then the evaluation of any function call
  $f(\mathbf{x})$ terminates.
\end{corollary}

\begin{fact}\label{fact2}
  For any equation $f(\mathbf{x}) = \expr{}$, if $\mathbf{x}$ contains only
  constants, then $\expr{}$ cannot include any function calls to $f$.
\end{fact}

Additionally, we introduce an assumption about the structure of recursion.

\begin{assumption}\label{assumption4}
  For every equation $(f(\mathbf{x}) = \expr) \in \mathcal{E}$, every recursive
  function call $f(\mathbf{y}) \in \expr$ satisfies the
  following:
  \begin{itemize}
    \item Each $y_{i}$ is either $x_{i} - c_{i}$ or $c_{i}$ for some constant
          $c_{i}$.
    \item There exists $i$ such that $y_{i} = x_{i} - c_{i}$ for some
          $c_{i} > 0$.
  \end{itemize}
\end{assumption}

Finally, we assume a particular order of evaluation for function calls using the
equations in $\mathcal{E}$. Specifically, we assume that base cases are
considered before the recursive definition. The exact order in which base cases
are considered is immaterial.

\begin{assumption}
  When multiple equations in $\mathcal{E}$ match a function call
  $f(\mathbf{x})$, preference is given to an equation with the most constants on
  its left-hand side.
\end{assumption}

With the observations and assumptions mentioned above, we prove the following
theorem.

\begin{theorem}[Termination]\label{thm:halting}
  Let $f$ be an $n$-ary function in $\mathcal{E}$ and
  $\mathbf{x} \in \mathbb{N}_{0}^{n}$. Then the evaluation of $f(\mathbf{x})$
  terminates.
\end{theorem}

For readability, we divide the proof into several lemmas of increasing
generality.

\begin{lemma}\label{lemma:oneunary}
  Assume that $\mathcal{E}$ consists of just \emph{one unary} function $f$. Then
  the evaluation of a function call $f(x)$ terminates for any
  $x \in \mathbb{N}_{0}$.
\end{lemma}
\begin{proof}
  If $f(x)$ is captured by a base case, then its evaluation terminates by
  \cref{fact,fact2}. If $f$ is not recursive, the evaluation of
  $f(x)$ terminates by \cref{fact}.

  Otherwise, let $f(y)$ be an arbitrary function call on the right-hand side of
  the definition of $f(x)$. If $y$ is a constant, then there is a base case for
  $f(y)$. Otherwise, let $y = x - c$ for some $c > 0$. Then there exists
  $k \in \mathbb{N}_{0}$ such that $0 \le x - kc \le c-1$. So, after $k$
  iterations, the sequence of function calls $f(x), f(x-c), f(x-2c),\dots$ will
  be captured by the base case $f(x \mod c)$.
\end{proof}

\begin{lemma}\label{lemma:onefunction}
  Generalising \cref{lemma:oneunary}, let $\mathcal{E}$ be a set of equations
  for \emph{one} $n$-ary function $f$ for some $n \ge 1$. Then the evaluation of
  $f(\mathbf{x})$ terminates for any $\mathbf{x} \in \mathbb{N}_{0}^{n}$.
\end{lemma}
\begin{proof}
  If $f$ is non-recursive, the evaluation of $f(\mathbf{x})$ terminates by
  previous arguments. We proceed by induction on $n$, with the base case of
  $n=1$ handled by \cref{lemma:oneunary}. Assume that $n > 1$. Any base case of
  $f$ can be seen as a function of arity $n-1$, since one of the parameters is
  fixed. Thus, the evaluation of any base case terminates by the inductive
  hypothesis. It remains to show that the evaluation of the recursive equation
  for $f$ terminates, but that follows from \cref{assumption3}.
\end{proof}

\begin{proof}[Proof of \cref{thm:halting}]
  We proceed by induction on the number of functions $n$. The base case of $n=1$
  is handled by \cref{lemma:onefunction}. Let ${(f_{i})}_{i=1}^{n}$ be some
  topological ordering of these $n > 1$ functions. If $f = f_{j}$ for $j < n$,
  then the evaluation of $f(\mathbf{x})$ terminates by the inductive hypothesis
  since $f_{j}$ cannot call $f_{n}$ by \cref{assumption2}. Using the inductive
  hypothesis that all function calls to $f_{j}$ (with $j < n$) terminate, the
  proof proceeds similarly to the Proof of \cref{lemma:onefunction}.
\end{proof}

\bibliography{paper}

\end{document}
